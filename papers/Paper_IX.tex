% ============================================================
% Paper IX — Falsifiability and Testable Predictions
% ============================================================

\documentclass[11pt]{article}

\usepackage{amsmath,amssymb}
\usepackage{graphicx}
\usepackage{hyperref}

% ============================================================
% Bibliography backend: biblatex + biber
% ============================================================
\usepackage[backend=biber,style=numeric,sorting=none]{biblatex}
\addbibresource{../shared/references.bib}

\usepackage{amsmath,amssymb}
% --- Graphics / TikZ ---
\usepackage{tikz}
\usetikzlibrary{arrows.meta,calc,decorations.pathreplacing}
% shared/macros.tex
\providecommand{\Afive}{A_{5}}
\providecommand{\golden}{\varphi} % optional standard name for \varphi
\providecommand{\phig}{\varphi}
\providecommand{\dd}{\mathrm{d}}
\providecommand{\ii}{\mathrm{i}}
\providecommand{\ee}{\mathrm{e}}
\providecommand{\Tr}{\mathrm{Tr}}
\providecommand{\diag}{\mathrm{diag}}
\providecommand{\SU}{\mathrm{SU}}
\providecommand{\U}{\mathrm{U}}


% ============================================================
% GPP_DRAFT_SUPPRESS_WARNINGS
% Draft-mode suppression of undefined reference warnings.
% Toggle: set \GPPdrafttrue (on) / \GPPdraftfalse (off)
% ============================================================
\newif\ifGPPdraft
\GPPdrafttrue

\makeatletter
\ifGPPdraft
  % Suppress only warning emissions (keeps errors, overfull boxes, etc.)
  \def\@latex@warning@no@line#1{}
  \def\@latex@warning#1{}
\fi
\makeatother



\begin{document}

\begin{center}
{\Large \bf Null Hypotheses, Falsifiability, and Testable Predictions}\\[0.4em]
{\normalsize Paper IX}
\end{center}

\vspace{1em}

\begin{abstract}
We formulate explicit null hypotheses corresponding to the discrete flavor structure identified in previous papers and define objective rejection criteria. We distinguish descriptive regularities from predictive claims and specify short-, intermediate-, and long-term tests capable of falsifying the framework. The purpose of this paper is not to extend the model, but to constrain it.
\end{abstract}

% ------------------------------------------------------------
\section{Purpose and scope}
\label{sec:scope}

The preceding papers have demonstrated reproducible structure in Standard Model masses and mixing phases under a discrete logarithmic analysis. This paper addresses a distinct question: \emph{what would it mean for this framework to be wrong?}

We therefore articulate null hypotheses, define quantitative rejection conditions, and enumerate predictions that can be evaluated independently of interpretation.

% ------------------------------------------------------------
\section{Primary null hypothesis}
\label{sec:null1}

\textbf{Null hypothesis H$_0^{(1)}$:}  
The observed alignment of fermion masses and mixing phases with discrete logarithmic structure is consistent with random assignment once anchoring, scan bounds, and tolerance effects are properly accounted for.

Rejection of H$_0^{(1)}$ requires demonstrating that:
\begin{enumerate}
\item solution multiplicities remain sparse after anchoring,
\item integer assignments are stable under bounded perturbations,
\item and cross-sector propagation (quark $\rightarrow$ lepton) occurs without refitting.
\end{enumerate}

Failure of any condition reinstates H$_0^{(1)}$.

% ------------------------------------------------------------
\section{Secondary null hypothesis: overfitting}
\label{sec:null2}

\textbf{Null hypothesis H$_0^{(2)}$:}  
The framework exhibits effective overfitting due to unconstrained integer freedom.

This null is rejected only if:
\begin{itemize}
\item integer solutions collapse to unique values after reference anchoring,
\item degeneracies do not reappear under expanded scan domains,
\item and equivalent predictive performance is not achieved by generic integer lattices.
\end{itemize}

All solution multiplicities are therefore reported explicitly in prior tables.

% ------------------------------------------------------------
\section{Immediate empirical tests}
\label{sec:immediate}

The following tests can be performed immediately using existing data:

\begin{enumerate}
\item Re-running all scans using updated PDG mass values and CKM fits.
\item Repeating PMNS phase extraction with alternative global-fit inputs.
\item Randomizing reference anchoring to verify loss of structure.
\end{enumerate}

Each test yields a binary outcome: preservation or collapse of structure.

% ------------------------------------------------------------
\section{Short-term predictions}
\label{sec:short}

Within the current experimental horizon, the framework predicts:

\begin{itemize}
\item CP phases must remain close to discrete predicted angles under refined measurements.
\item New global fits should not introduce additional viable integer solutions within tolerance.
\end{itemize}

Deviation beyond stated tolerances falsifies the framework without ambiguity.

% ------------------------------------------------------------
\section{Longer-term projections}
\label{sec:long}

If the observed structure reflects an underlying organizing principle, it may extend to:
\begin{enumerate}
\item refined neutrino mass ordering constraints,
\item relations between flavor parameters and geometric invariants,
\item or additional discrete regularities at higher precision.
\end{enumerate}

These are explicitly identified as \emph{speculative extensions} and are not required for validation.

% ------------------------------------------------------------
\section{Failure modes}
\label{sec:failure}

The framework is considered falsified if:
\begin{itemize}
\item solution multiplicities proliferate uncontrollably,
\item predicted CP phases drift with improved data,
\item or comparable structure arises generically from random lattices.
\end{itemize}

Each failure mode is empirically detectable.

% ------------------------------------------------------------
\section{Concluding remarks}
\label{sec:conclusion}

This paper establishes the empirical accountability of the discrete flavor framework. By formulating explicit null hypotheses and rejection criteria, we ensure that the results reported earlier remain subject to decisive experimental and statistical scrutiny.

If future data preserve the observed structure, the framework warrants deeper theoretical investigation. If not, it is conclusively ruled out.

\end{document}
