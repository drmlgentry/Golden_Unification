% ============================================================
% Paper VIII — Synthesis of Discrete Flavor Geometry
% ============================================================

\documentclass[11pt]{article}

\usepackage{amsmath,amssymb}
\usepackage{graphicx}
\usepackage{hyperref}

% ============================================================
% Bibliography backend: biblatex + biber
% ============================================================
\usepackage[backend=biber,style=numeric,sorting=none]{biblatex}
\addbibresource{../shared/references.bib}

\usepackage{amsmath,amssymb}
% --- Graphics / TikZ ---
\usepackage{tikz}
\usetikzlibrary{arrows.meta,calc,decorations.pathreplacing}
% shared/macros.tex
\providecommand{\Afive}{A_{5}}
\providecommand{\golden}{\varphi} % optional standard name for \varphi
\providecommand{\phig}{\varphi}
\providecommand{\dd}{\mathrm{d}}
\providecommand{\ii}{\mathrm{i}}
\providecommand{\ee}{\mathrm{e}}
\providecommand{\Tr}{\mathrm{Tr}}
\providecommand{\diag}{\mathrm{diag}}
\providecommand{\SU}{\mathrm{SU}}
\providecommand{\U}{\mathrm{U}}


% ============================================================
% GPP_DRAFT_SUPPRESS_WARNINGS
% Draft-mode suppression of undefined reference warnings.
% Toggle: set \GPPdrafttrue (on) / \GPPdraftfalse (off)
% ============================================================
\newif\ifGPPdraft
\GPPdrafttrue

\makeatletter
\ifGPPdraft
  % Suppress only warning emissions (keeps errors, overfull boxes, etc.)
  \def\@latex@warning@no@line#1{}
  \def\@latex@warning#1{}
\fi
\makeatother



\begin{document}

\begin{center}
{\Large \bf Synthesis of Discrete Flavor Geometry and Empirical Structure}\\[0.4em]
{\normalsize Paper VIII}
\end{center}

\vspace{1em}

\begin{abstract}
This paper synthesizes the empirical and mathematical results developed across the preceding works into a coherent framework. We consolidate evidence that Standard Model fermion masses and mixing phases exhibit stable alignment with a discrete logarithmic organization when expressed relative to a fixed reference scale. Anchoring procedures, bounded integer scans, and cross-sector consistency collectively indicate that the observed structure is neither arbitrary nor generically proliferative. We distinguish established results from conjectural interpretation and delineate the precise empirical content of the framework without extending beyond demonstrated scope.
\end{abstract}

% ------------------------------------------------------------
\section{Overview of the program}
\label{sec:overview}

The preceding papers have developed and tested a phenomenological framework in which fermion masses and mixing parameters are analyzed in logarithmic space relative to a chosen reference scale. The central methodological premise is that regularity, if present, should manifest more transparently in dimensionless logarithmic variables than in raw mass values.

This paper does not introduce new calculations. Instead, it consolidates results, clarifies logical dependencies, and articulates what has and has not been established.

% ------------------------------------------------------------
\section{Summary of established results}
\label{sec:established}

The following statements summarize results demonstrated explicitly through numerical analysis and reproducible code:

\begin{enumerate}
\item When expressed logarithmically relative to a reference mass, the Standard Model fermion spectrum exhibits structured alignment inconsistent with a featureless distribution.
\item Fixing the electron as an anchoring reference removes continuous degeneracies and yields sparse integer solutions within bounded search regions.
\item Best-fit integer triples remain stable under moderate variations of scan bounds and tolerance thresholds.
\item The CKM CP-violating phase is recovered with high numerical precision from a single predicted geometric angle without free fitting.
\item Extension of the same procedure to the leptonic sector yields consistent PMNS CP phases using identical methodology.
\end{enumerate}

Each result above is supported by explicit scan outputs and deterministic scripts included in the public repository.

% ------------------------------------------------------------
\section{Cross-sector coherence}
\label{sec:coherence}

A notable feature of the framework is that mass spectra and mixing phases are treated on equal footing. The same discrete geometric input underlies both sectors, and no additional sector-specific parameters are introduced.

The recovery of both CKM and PMNS phases from a common structural origin suggests that the observed organization is not confined to a single phenomenological domain. This coherence substantially strengthens the interpretation of the results as structural rather than coincidental.

% ------------------------------------------------------------
\section{Non-claims and limitations}
\label{sec:limitations}

It is essential to emphasize what is \emph{not} claimed by this work. The framework does not provide:
\begin{itemize}
\item a dynamical mechanism for mass generation,
\item a modification of the Standard Model Lagrangian,
\item or a unique ultraviolet completion.
\end{itemize}

The results are phenomenological and structural. They identify regularity in observed parameters but do not attempt to derive those parameters from first principles. Any deeper interpretation must be justified independently.

% ------------------------------------------------------------
\section{Interpretive posture}
\label{sec:interpretation}

The appearance of discrete organization across masses and mixing phases motivates the hypothesis that flavor parameters encode geometric information not manifest at the level of low-energy dynamics. In this view, flavor structure may reflect emergent constraints rather than direct symmetries.

This interpretation is offered as a motivation for further investigation rather than as a conclusion. The empirical content of the framework stands independently of any geometric narrative.

% ------------------------------------------------------------
\section{Relation to falsifiability}
\label{sec:falsifiability}

This synthesis prepares the ground for explicit falsifiability analysis. The stability, sparsity, and cross-sector propagation of solutions identified here enable precise null hypotheses and rejection criteria, which are formulated in the following paper.

By separating synthesis from prediction and falsification, we aim to maintain a clear epistemic boundary between established results and forward-looking tests.

% ------------------------------------------------------------
\section{Conclusion}
\label{sec:conclusion}

Taken together, the results consolidated in this paper demonstrate that Standard Model flavor parameters exhibit nontrivial discrete structure when analyzed in logarithmic space with appropriate anchoring. The framework is empirically grounded, methodologically transparent, and reproducible.

Whether this structure reflects deeper geometric principles or constitutes an unexpected phenomenological regularity remains an open question. The answer must be decided by falsifiable tests rather than interpretation, which we address next.

\end{document}

