% ============================================================
% Paper VII — Geometric Scaling, Thresholds, and Cosmology
% ============================================================

\section{Paper VII: Geometric Scaling Beyond Flavor}
\label{paper:VII}

\subsection{Abstract}

We examine whether the discrete logarithmic structure identified in the Standard Model flavor sector extends consistently to higher mass scales and cosmological quantities. Without introducing new dynamics or modifying known field-theoretic mechanisms, we investigate whether characteristic thresholds—including electroweak, heavy-quark, and Planck-adjacent scales—exhibit alignment with the same anchored lattice organization used to describe fermion masses and mixing phases. Emphasis is placed on dimensional consistency, falsifiability, and the distinction between structural regularity and numerical coincidence.

\subsection{Motivation and scope}

Papers III–VI established that fermion masses and mixing observables admit a compact description in terms of discrete logarithmic coordinates anchored to a reference state. A natural question is whether this organization is confined to flavor physics or reflects a broader regularity in known physical scales.

The goal of the present paper is deliberately modest. We do not propose a theory of quantum gravity, nor do we derive cosmological parameters from first principles. Instead, we ask a sharply defined phenomenological question:

\begin{quote}
Do known non-flavor mass scales occupy positions in logarithmic mass space that are compatible with the same anchored lattice used for Standard Model masses?
\end{quote}

A negative answer would strongly constrain the interpretation of the lattice as a physically meaningful structure.

\subsection{Logarithmic scaling framework}

We consider mass scales $M$ expressed relative to a fixed reference mass $m_{\mathrm{ref}}$ via
\begin{equation}
\Delta(M) \equiv \log_{\varphi}\!\left(\frac{M}{m_{\mathrm{ref}}}\right),
\end{equation}
where $\varphi$ denotes the golden ratio. This definition mirrors the construction used in the flavor sector but is applied here purely as a diagnostic coordinate.

Importantly, no assumption is made that $\varphi$ plays a dynamical role at high energies. Its use here reflects only its role as a convenient base for comparing logarithmic separations already identified in lower-energy data.

\subsection{Electroweak and heavy-particle thresholds}

We first consider several well-established thresholds:
\begin{itemize}
\item the Higgs vacuum expectation value,
\item the electroweak scale $m_W$,
\item the top-quark mass,
\item representative heavy-quark thresholds.
\end{itemize}

When expressed in $\Delta(M)$ coordinates relative to the same reference used in Paper III, these scales occupy positions that are not uniformly distributed. Instead, they cluster near integer or half-integer offsets relative to the fermion lattice spacing.

We stress that this observation is descriptive rather than explanatory. No tuning is performed, and no parameters are fit. The question addressed is solely whether these scales are grossly incompatible with the established lattice spacing. They are not.

\subsection{Planck scale and dimensional separation}

The Planck mass $M_{\mathrm{Pl}}$ lies far outside the Standard Model range, yet it provides a useful stress test of the framework. Expressed in logarithmic units,
\begin{equation}
\Delta(M_{\mathrm{Pl}}) \gg \Delta(m_t),
\end{equation}
as expected.

Remarkably, however, the separation between $M_{\mathrm{Pl}}$ and electroweak-scale quantities remains commensurate with an extension of the same logarithmic spacing, up to order-unity deviations. Given the enormous dynamic range involved, this compatibility is nontrivial but should not be over-interpreted.

In particular, we do \emph{not} claim that the Planck scale is predicted by the lattice, only that it does not falsify it.

\subsection{Cosmological scales}

Cosmological parameters such as the present Hubble scale or vacuum energy density introduce additional subtleties, including scheme dependence and epoch dependence. For this reason, we treat them only qualitatively.

When mapped into logarithmic mass units using standard conventions, these scales fall far below the fermion sector but again do not violate the spacing pattern by many orders of magnitude. Whether this reflects a meaningful geometric relation or merely dimensional coincidence remains open.

\subsection{Null tests and controls}

To assess whether the observed compatibility could arise generically, we compare the empirical scales above against randomized logarithmic spectra with matched dynamic range. In such ensembles, comparable alignment occurs with low but non-negligible probability.

This reinforces an important conclusion: the present results are suggestive but not decisive. They motivate further study but do not constitute evidence of unification or new symmetry.

\subsection{Interpretation and limits}

The most conservative interpretation of the results in this paper is that the discrete logarithmic lattice identified in flavor physics does not immediately fail when confronted with known non-flavor scales.

This leaves open several possibilities:
\begin{enumerate}
\item the lattice reflects a genuine organizing principle that spans multiple sectors,
\item it captures a phenomenological regularity without deep dynamical origin,
\item or it arises from coincidental alignments amplified by logarithmic compression.
\end{enumerate}

Discriminating among these possibilities requires independent predictions, addressed in Paper VIII.

\subsection{Summary}

We have examined whether established mass scales beyond the Standard Model flavor sector are grossly incompatible with the anchored logarithmic lattice framework developed in earlier papers. They are not.

No new parameters were introduced, no fits were performed, and no claims of derivation were made. The results serve as a consistency check and a boundary condition for interpretation rather than as evidence of new physics.

The logical consequences of this observation, and the criteria under which the program should be considered falsified, are addressed in the final paper of this series.

