\begin{figure}[t]
\centering
\subsection{Origami vertex inversion as an orientation reversal: a holonomy toy model}
\label{subsec:origami_inversion_holonomy_figA}

We record a minimal (non-dynamical) geometric toy model to connect the observed
``pop-through'' inversion of a locally pentagonal origami patch to the behavior of
Wilson-loop phases. The key point is topological: if the inversion reverses the local
normal while preserving the cyclic order of edges in the plane of the patch, then the
associated loop orientation is reversed, mapping $\gamma \mapsto \gamma^{-1}$.

In a $U(1)$ toy gauge field, the Wilson loop is
\begin{equation}
W[\gamma] \;=\; \exp\!\left(i \oint_{\gamma} A\right)
\;=\; \exp\!\left(i\,\Phi[\gamma]\right),
\end{equation}
so orientation reversal gives $\Phi[\gamma^{-1}] = -\Phi[\gamma]$ and hence
\begin{equation}
W[\gamma^{-1}] \;=\; \exp\!\left(-i\,\Phi[\gamma]\right).
\end{equation}
This is the schematic mechanism by which CP-sensitive phases can change sign while
purely ``mass-like'' assignments (which do not depend on loop orientation) remain unchanged.
We emphasize that this is a kinematic/topological illustration; it is not yet a microscopic
derivation of CKM/PMNS phases.
\caption{%
\\textbf{Schematic holonomy loop used to illustrate phase sensitivity under orientation reversal.}
The diagram shows an abstract closed path and its transformation under inversion. The construction is illustrative and does not represent a physical trajectory or dynamical process.
}
\end{figure}