

\subsection{Origami vertex inversion as an orientation reversal: a holonomy toy model}
\label{subsec:origami_inversion_holonomy_figA}

We record a minimal (non-dynamical) geometric toy model to connect the observed
``pop-through'' inversion of a locally pentagonal origami patch to the behavior of
Wilson-loop phases. The key point is topological: if the inversion reverses the local
normal while preserving the cyclic order of edges in the plane of the patch, then the
associated loop orientation is reversed, mapping $\gamma \mapsto \gamma^{-1}$.

In a $U(1)$ toy gauge field, the Wilson loop is
\begin{equation}
W[\gamma] \;=\; \exp\!\left(i \oint_{\gamma} A\right)
\;=\; \exp\!\left(i\,\Phi[\gamma]\right),
\end{equation}
so orientation reversal gives $\Phi[\gamma^{-1}] = -\Phi[\gamma]$ and hence
\begin{equation}
W[\gamma^{-1}] \;=\; \exp\!\left(-i\,\Phi[\gamma]\right).
\end{equation}
This is the schematic mechanism by which CP-sensitive phases can change sign while
purely ``mass-like'' assignments (which do not depend on loop orientation) remain unchanged.
We emphasize that this is a kinematic/topological illustration; it is not yet a microscopic
derivation of CKM/PMNS phases.

\begin{figure}[t]
\centering
\begin{tikzpicture}[scale=1.0, >=Latex]

\coordinate (C) at (0,0);
\foreach \k in {0,72,144,216,288}{
  \coordinate (P\k) at ({1.6*cos(\k)},{1.6*sin(\k)});
}

\foreach \k/\kp in {0/72,72/144,144/216,216/288,288/0}{
  \draw[thick] (C) -- (P\k) -- (P\kp) -- cycle;
}

\draw[thick] (P0) -- (P72) -- (P144) -- (P216) -- (P288) -- cycle;

\draw[very thick, -{Latex[length=3mm]}]
  ($(P0)!0.55!(P72)$) arc [start angle=18, end angle=342, radius=1.05];

\node at (0,-2.2) {\small (a) Local patch with loop $\gamma$};

\draw[->, very thick] (2.6,0.0) -- (4.2,0.0);
\node at (3.4,0.35) {\small inversion};
\node at (3.4,-0.35) {\small $z \mapsto -z$};

\begin{scope}[shift={(6.2,0)}]
\coordinate (C2) at (0,0);
\foreach \k in {0,72,144,216,288}{
  \coordinate (Q\k) at ({1.6*cos(\k)},{1.6*sin(\k)});
}
\foreach \k/\kp in {0/72,72/144,144/216,216/288,288/0}{
  \draw[thick] (C2) -- (Q\k) -- (Q\kp) -- cycle;
}
\draw[thick] (Q0) -- (Q72) -- (Q144) -- (Q216) -- (Q288) -- cycle;

\draw[very thick, -{Latex[length=3mm]}]
  ($(Q0)!0.55!(Q72)$) arc [start angle=18, end angle=-342, radius=1.05];

\node at (0,-2.2) {\small (b) Orientation-reversed loop $\gamma^{-1}$};
\node[align=center] at (0,2.15) {\small $W[\gamma]=e^{i\Phi[\gamma]}$\\[-1mm]
\small $W[\gamma^{-1}]=e^{-i\Phi[\gamma]}$};
\end{scope}

\end{tikzpicture}
\caption{\textbf{Schematic holonomy loop used to illustrate phase sensitivity under orientation reversal.} The diagram shows an abstract closed path and its transformation under inversion. The construction is illustrative and does not represent a physical trajectory or dynamical process.}
The diagram shows an abstract closed path and its transformation under inversion. The construction is illustrative and does not represent a physical trajectory or dynamical process.}
A loop $\gamma$ on a discrete local patch acquires a holonomy phase $W[\gamma]=\exp(i\Phi[\gamma])$.
A vertex inversion along the local normal (origami ``pop-through'') reverses loop orientation without changing adjacency data, mapping $\gamma\mapsto\gamma^{-1}$ and hence $\Phi\mapsto -\Phi$.
This provides a minimal schematic for why mass-like observables can remain invariant while CP-sensitive phases (through $\sin\delta$) change under an orientation reversal.
}
\label{fig:holonomy_inversion}
\end{figure}