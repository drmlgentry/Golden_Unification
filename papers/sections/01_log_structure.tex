\section{Discrete Logarithmic Structure in Flavor Data}
\label{sec:log_structure}

Fermion masses in the Standard Model span more than twelve orders of magnitude, from
sub-eV neutrino scales to the top-quark mass near the electroweak scale. When represented
on linear axes, this hierarchy appears irregular and weakly structured. However, when
expressed in logarithmic coordinates, mass ratios become additive quantities, and potential
organizational regularities---if present---are rendered more transparent.

This observation is not new in itself: renormalization-group flow, dimensional transmutation,
and hierarchical symmetry breaking naturally generate exponential relations among physical
scales. What is less commonly emphasized is that, when logarithmic mass ratios are compared
directly across particle species under fixed conventions, the observed spectrum can exhibit
non-random clustering suggestive of discrete organization.

In this work we therefore adopt logarithmic coordinates as the natural arena in which to
examine structural regularities in flavor data. Writing each mass as
\begin{equation}
m_i \;=\; m_{\mathrm{ref}} \, e^{\Delta_i},
\end{equation}
the dimensionless quantities $\Delta_i$ encode relative mass information independently of
the choice of reference scale $m_{\mathrm{ref}}$. Throughout, all comparisons are performed
in such dimensionless logarithmic variables.

The central empirical hypothesis explored here is that measured fermion masses and associated
flavor observables cluster near a discrete subset of logarithmic values once a reference
convention is fixed. In the construction used in this paper, the relevant discreteness is
expressed in base-$\varphi$ logarithms with quarter-integer spacing (with half-integer and
integer steps appearing as special cases; see Sec.~\ref{sec:lattice}). Importantly, this
hypothesis does not posit a mass-generation mechanism, an underlying symmetry, or a dynamical
explanation. Rather, it asserts only the existence---or absence---of a reproducible regularity
in the data under explicitly declared constructions.

Our objective is therefore deliberately limited. We do not assume that the observed structure
reflects a fundamental law, nor do we infer ultraviolet dynamics from it. Instead, we document
the regularity itself using a transparent, auditable framework and specify clear falsification
criteria. In particular, any claimed clustering must persist under controlled perturbations of
input data, anchoring choices, and bounded scan protocols. Failure to do so constitutes evidence
against the organizing principle.

Throughout, the emphasis is on reproducibility, explicit assumptions, and sharply delimited
interpretation. Questions of deeper meaning---if any---are deferred until after the empirical
structure has been established or ruled out.