

\subsection{\texorpdfstring{$A_{5}$}{A5} as icosahedral/dodecahedral rotational symmetry: an intuitive axis picture}
\label{subsec:A5_axes_picture_figB}

For reader orientation, we record the standard geometric identification:
the alternating group $A_{5}$ is isomorphic to the orientation-preserving rotational symmetry group of the
icosahedron (equivalently, the dodecahedron). This gives an intuitive geometric meaning to phrases like
``$A_{5}$-structured'' or ``residual $Z_{5}$'': the relevant discrete structures are naturally organized around
the 5-fold, 3-fold, and 2-fold symmetry axes.

In holonomy language, discrete phase behavior can be associated with transport around loops linked with
these symmetry-defining axes. The figure below is schematic; it is not a full polyhedron rendering, but it
captures the group-theoretic content needed for a conservative narrative.

\begin{figure}[t]
\centering
\begin{tikzpicture}[scale=1.0, >=Latex]

\draw[thick] (0,0) circle (2.2);

\draw[very thick] (0,-2.2) -- (0,2.2);
\node[align=center] at (0,2.55) {\small 5-fold axis};

\draw[thick] ({-2.2*cos(30)},{-2.2*sin(30)}) -- ({2.2*cos(30)},{2.2*sin(30)});
\node[align=center] at (2.6,1.3) {\small 3-fold};

\draw[thick] ({-2.2*cos(90)},{-2.2*sin(90)}) -- ({2.2*cos(90)},{2.2*sin(90)});
\node[align=center] at (2.8,0.0) {\small 2-fold};

\draw[very thick, -{Latex[length=3mm]}] (0.9,0.0) arc[start angle=0,end angle=330,radius=0.9];
\node at (0.0,-0.9) {\small loop around 5-fold axis};

\node[align=center] at (0,-2.75) {\small $A_{5} \cong \mathrm{Rot}(\text{icosahedron/dodecahedron})$};

\end{tikzpicture}
\caption{\textbf{Schematic representation of \(A_{5}\) symmetry axes used for geometric visualization.} The figure illustrates the relative orientations of the discrete axes employed in the construction discussed in the text. No metric or dynamical information is encoded; the diagram serves only as a geometric reference for the labeling conventions.}\) symmetry axes used for geometric visualization.}
The figure illustrates the relative orientations of the discrete axes employed in the construction discussed in the text. No metric or dynamical information is encoded; the diagram serves only as a geometric reference for the labeling conventions.}\) structure (schematic).}
The group \(A_{5}\) is isomorphic to the orientation-preserving rotational symmetry group of the icosahedron/dodecahedron.
This provides an intuitive geometric interpretation for ``\(A_{5}\)-structured'' residual symmetries:
discrete 5-fold, 3-fold, and 2-fold axes organize admissible identifications.
In holonomy language, transporting data around loops linked with these axes can yield phase behavior sensitive to orientation.
}
\label{fig:A5_axes}
\end{figure}