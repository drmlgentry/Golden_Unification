\section{Interpretation, context, and limits}
\label{sec:discussion}

The results reported in Secs.~\ref{sec:results} and \ref{sec:integer_fits} establish that,
under a fixed logarithmic encoding and anchoring convention, measured fermion masses cluster
non-randomly near a discrete lattice in logarithmic space. This section clarifies what may
and may not be inferred from that observation.

\subsection{What is established}

The analysis demonstrates three empirical facts.

First, once an anchor is fixed, the logarithmic offsets of fermion masses admit integer or
half-integer assignments with bounded residuals that are small compared to the overall
hierarchy scale. These assignments are obtained by a deterministic scan with no hand-tuning.

Second, the resulting fits are structurally constrained: multiplicities remain controlled,
and the residual structure is stable under small perturbations of input values within their
quoted uncertainties.

Third, the pattern persists across fermion sectors under a single declared convention.
This persistence is nontrivial: relaxing the anchoring or allowing additional free parameters
rapidly degrades predictivity rather than improving it.

Together, these points establish the existence of a reproducible regularity in the data
within the stated framework.

\subsection{What is not claimed}

No dynamical mechanism for mass generation is proposed here. In particular, the lattice
structure is not derived from a symmetry principle, a potential, or ultraviolet dynamics.
The integers that appear in the fits are treated as descriptive labels, not quantum numbers.

Likewise, the analysis does not claim statistical optimality in a global sense. The lattice
is not asserted to be the best possible parametrization among all conceivable models, only
that it exhibits non-random organization under explicitly fixed rules.

Finally, no claim is made that the observed structure must persist beyond the currently
measured fermion spectrum. The framework is falsifiable, and future data may invalidate
it.

\subsection{Relation to existing approaches}

Hierarchical fermion masses are often modeled using continuous flavor symmetries, Froggatt--Nielsen
mechanisms, or renormalization-group effects. These approaches typically emphasize exponential
hierarchies but do not require discrete alignment in logarithmic space.

The present construction is orthogonal to such models. It neither assumes nor excludes them.
Instead, it isolates a purely empirical question: whether the observed spectrum itself
exhibits discrete organization when expressed in the appropriate variables.

\subsection{Interpretive caution}

Because the lattice is defined only after an anchoring convention is declared, interpretive
overreach must be avoided. The anchoring choice is not optimized, scanned, or tuned, and
different choices would lead to different integer assignments.

For this reason, the results should be read as conditional statements: \emph{given} the
declared encoding and anchor, the data exhibit the reported structure. Any physical
interpretation must respect that conditionality.

\subsection{Forward link}

The remaining sections extend the framework to mixing observables and CP-violating phases
under the same audit principles. Methodological details of the scan protocol, encoding
choices, and reproducibility machinery are documented separately in the companion methods
paper.
