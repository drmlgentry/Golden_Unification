\section{Mixing observables and CP-sensitive invariants}
\label{sec:mixing}

The lattice framework introduced above applies directly to fermion masses. Mixing angles
and CP-violating phases require a distinct treatment, but are incorporated here using the
same guiding principles: fixed conventions, rephasing invariance, and auditability.

\subsection{Mixing matrices and conventions}

We adopt the standard three-generation parameterization of quark and lepton mixing matrices,
with three mixing angles and one Dirac CP phase. All conventions are fixed once and held
constant throughout the analysis.

No attempt is made to optimize phase conventions or rotate away apparent structure. The goal
is not to maximize alignment, but to test whether nontrivial regularities persist under
standard choices.

\subsection{Rephasing-invariant CP measure}

To quantify CP violation independently of basis choices, we use the Jarlskog invariant $J$,
defined by
\begin{equation}
J \;\equiv\; \mathrm{Im}\!\left(
V_{ij} V_{kl} V_{il}^\ast V_{kj}^\ast
\right),
\label{eq:jarlskog}
\end{equation}
for any distinct $i \neq k$ and $j \neq l$, where $V$ denotes the relevant mixing matrix.

The magnitude of $J$ provides a basis-independent measure of CP violation, while its sign
encodes orientation information once a convention is fixed.

\subsection{Role of invariants in the lattice framework}

Within the present framework, mixing observables are not fit independently of masses.
Instead, they are treated as diagnostic extensions: once a lattice structure is established
for masses, mixing parameters test whether related regularities appear in phase space.

The Jarlskog invariant is particularly useful in this role. It collapses phase information
into a single invariant quantity that can be compared across models and conventions.

\subsection{Scope of application}

This paper does not propose a lattice structure for mixing angles themselves. Rather, it
tests whether observed CP-sensitive invariants are consistent with the same degree of
constraint and stability observed in the mass sector.

Numerical verification of mixing phases and their consistency with lattice-aligned inputs
is presented in the subsequent verification and results-summary sections.