\section{Logarithmic Lattice and Anchoring Conventions}
\label{sec:lattice}

To test whether fermion masses admit a discrete organization in logarithmic space,
we must specify both the coordinate representation in which comparisons are made
and a fixed convention that removes trivial global rescalings. This section defines
the anchored lattice construction used throughout the paper. The construction is
methodological rather than interpretive: once fixed, it is held constant across all
analyses reported here.

\subsection{Logarithmic coordinates and reference normalization}

Given a set of measured masses $\{m_i\}$ and a reference mass $m_{\mathrm{ref}}$,
we define logarithmic offsets
\begin{equation}
\Delta_i \;\equiv\; \ln\!\left(\frac{m_i}{m_{\mathrm{ref}}}\right).
\label{eq:log_offsets}
\end{equation}
This change of variables removes overall scale dependence and converts
multiplicative hierarchies into additive structure. No assumptions about dynamics,
symmetry, or discreteness are encoded in Eq.~\eqref{eq:log_offsets}.

Throughout this paper we take $m_{\mathrm{ref}} = m_e$ (the electron mass) and
express results in logarithms base $\varphi = (1+\sqrt{5})/2$ for compactness.

\subsection{Discrete encoding map}

We introduce a discrete index constructed from integer data. Let
$(a,b,c)\in\mathbb{Z}^3$ and define the linear functional
\begin{equation}
q(a,b,c) \;=\; 8a + 15b + 24c.
\label{eq:q_def}
\end{equation}
The associated mass-ratio hypothesis is
\begin{equation}
\frac{m(a,b,c)}{m_e} \;=\; \varphi^{\,q(a,b,c)/4}.
\label{eq:mass_map}
\end{equation}
Equations~\eqref{eq:q_def}–\eqref{eq:mass_map} introduce no continuous fitting
parameters. The only degrees of freedom are integer assignments within declared
scan bounds. The effective lattice spacing in $\log_\varphi m$ is therefore a
quarter-integer, with half-integer and integer steps appearing as special cases.

\paragraph{Terminology.}
For brevity, we refer to the fixed encoding map
(Eqs.~\eqref{eq:q_def}–\eqref{eq:mass_map}) together with the declared anchoring
rule below as the \emph{anchored lattice}. This term is used uniformly throughout
the paper and does not imply additional structure beyond what is stated explicitly.

\subsection{Anchoring convention}

The mapping in Eq.~\eqref{eq:mass_map} is invariant under uniform shifts
$q \mapsto q + \Delta q$, corresponding to a global rescaling of all masses.
Because the analysis is comparative, this redundancy must be removed.

We impose an anchor by fixing the electron as the reference state,
\begin{equation}
\Delta_e = 0 \quad \Longleftrightarrow \quad q_e = 0.
\label{eq:anchor}
\end{equation}
This convention fixes the origin of logarithmic mass space. It is declared once
and held fixed across all sectors; it is not optimized to improve fits.

\subsection{Bounded scan and residual definition}

For a given experimental mass $m_{\mathrm{exp}}$, we scan bounded integer domains
\begin{equation}
(a,b,c) \in [a_{\min},a_{\max}] \times [b_{\min},b_{\max}] \times [c_{\min},c_{\max}]
\label{eq:scan_bounds}
\end{equation}
and evaluate the signed logarithmic discrepancy
\begin{equation}
\delta(a,b,c) \;\equiv\;
\log_{\varphi}\!\left(\frac{m(a,b,c)}{m_{\mathrm{exp}}}\right).
\label{eq:signed_delta}
\end{equation}
When reporting fit quality, we use the absolute residual
\begin{equation}
\epsilon(a,b,c) \;\equiv\; |\delta(a,b,c)|,
\label{eq:epsilon_def}
\end{equation}
while retaining the sign of $\delta$ for diagnostic purposes.

\subsection{Multiplicity and auditability}

Because the encoding depends only on $q$, distinct integer triples inducing the
same $q$ are grouped when reporting solutions. We therefore report both
(i) best-fit residuals and (ii) multiplicities after deduplication by $q$.
All numerical tables included in the paper are generated by deterministic scripts
and inserted verbatim as repository artifacts. Any independent implementation
using the same inputs, anchoring convention, scan bounds, and residual definition
must reproduce the same outputs.
