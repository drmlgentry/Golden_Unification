

\section{Origami inversion as a non-Abelian holonomy toy model}
\label{sec:06d_origami_inversion_holonomy}

\paragraph{Purpose and scope.}
This section provides a controlled toy model for the intuitive ``pop-through'' (inversion) move:
a discrete change in an internal frame assignment around a closed cycle. The point is not to
claim a microscopic gauge theory, but to express the transformation using standard, rigorous
language---holonomy and Wilson loops---so that later extensions can be benchmarked against
established frameworks.

\subsection{Setup: a closed loop and a frame-valued connection}
Let $\mathcal{C}$ be a closed cycle (a loop) in the configuration/parameter space relevant to the
protocol under study. Choose a compact Lie group $G$ (e.g.\ $U(1)$ for an Abelian toy model or
$SU(2)$ for a minimal non-Abelian toy model), and let
$A$ be a $\mathfrak{g}$-valued connection one-form on $\mathcal{C}$.

The associated holonomy around $\gamma:\,[0,1]\to \mathcal{C}$ with $\gamma(0)=\gamma(1)$ is
\begin{equation}
\label{eq:06d_holonomy_def}
U[\gamma] \;\equiv\; \mathcal{P}\exp\!\left(i\oint_{\gamma} A\right)\in G,
\end{equation}
where $\mathcal{P}$ denotes path ordering.

The corresponding Wilson loop in a representation $\rho$ of $G$ is
\begin{equation}
W_{\rho}[\gamma] \;\equiv\; \mathrm{Tr}\,\rho\!\left(U[\gamma]\right),
\label{eq:06d_wilson_loop_def}
\end{equation}
which is gauge invariant.

\subsection{Abelian limit and why the non-Abelian model subsumes it}
If $G=U(1)$, then $\mathcal{P}$ is unnecessary and the holonomy reduces to a phase,
\begin{equation}
U_{U(1)}[\gamma] \;=\; \exp\!\left(i\oint_{\gamma} A\right),
\label{eq:06d_abelian_phase}
\end{equation}
so the Abelian construction is recovered as the commutative special case of
Eq.~\eqref{eq:06d_holonomy_def}.

In this sense, the non-Abelian formulation \emph{subsumes} the Abelian toy model:
all Abelian statements arise by restricting to commuting connections (or projecting to an
Abelian subgroup).

\subsection{A discrete ``pop-through'' as a holonomy class}
The ``inversion'' move is modeled as a discrete change in the holonomy class of the loop:
rather than tracking a continuously varying parameter, we track whether the protocol induces a
nontrivial group element $U[\gamma]$ (or its conjugacy class).

\paragraph{Conjugacy-class data.}
In a non-Abelian setting, the physically meaningful invariant of holonomy is not $U[\gamma]$
itself but its conjugacy class:
\begin{equation}
U[\gamma]\sim g\,U[\gamma]\,g^{-1}.
\label{eq:06d_conjugacy}
\end{equation}
Thus, a ``pop-through'' can be captured by a jump between conjugacy classes (or between
distinguished representatives) under a discrete re-identification of frames.

\subsection{Curvature/flux interpretation (Stokes-type intuition)}
When $\gamma=\partial \Sigma$ bounds a surface $\Sigma$ in an ambient space where $A$ extends,
one can interpret holonomy as a curvature/flux measure. In Abelian language,
\begin{equation}
\oint_{\gamma} A \;=\; \int_{\Sigma} F, \qquad F=dA,
\label{eq:06d_stokes_abelian}
\end{equation}
so the phase is controlled by the integrated curvature/field strength through $\Sigma$.

In the non-Abelian setting, the analogous statement is more subtle (surface ordering is needed),
but the operational interpretation remains: \emph{holonomy detects the integrated curvature content
enclosed by the loop}. This is precisely the correct formal device for modeling a discrete,
topological or frame-relabeling effect.

\subsection{Connection to discrete symmetry and an $\mathsf{A}_5$-type modulus}
A discrete symmetry such as $\mathsf{A}_5$ can enter at the level of \emph{allowed holonomy classes}
or \emph{allowed frame identifications}. Concretely, one may posit that the protocol restricts
$U[\gamma]$ to a finite subset of conjugacy classes compatible with an $\mathsf{A}_5$-structured
identification (e.g.\ dodecahedral/icosahedral residual structure). In that case, the modulus-like
parameter is not a continuous angle but a discrete (or discretized) holonomy datum, and the
inversion corresponds to selecting a different admissible class.

\paragraph{What is claimed here.}
This section claims only the following:
\begin{enumerate}
\item The inversion can be represented rigorously as a holonomy/Wilson-loop observable.
\item The non-Abelian formulation automatically contains the Abelian phase model as a special case.
\item If an $\mathsf{A}_5$ (dodecahedral) identification is present, it can be imposed as a
constraint on admissible holonomy classes.
\end{enumerate}
Any phenomenological matching (e.g.\ to CKM/PMNS phases or to lattice-encoded logarithmic structure)
must be treated as an additional hypothesis and tested with the same reproducibility and null
discipline used elsewhere in this manuscript.
