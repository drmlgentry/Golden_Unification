% Paper_III.tex — Anchored integer-lattice fits and results
% =========================================================

\documentclass[12pt]{article}
\usepackage{amsmath,amssymb,amsthm}
\usepackage{geometry}
\usepackage{hyperref}
\usepackage{graphicx}
\geometry{margin=1in}

% Shared macros (defines \golden, \Afive, etc., if you add them there)
% ============================================================
% Bibliography backend: biblatex + biber
% ============================================================
\usepackage[backend=biber,style=numeric,sorting=none]{biblatex}
\addbibresource{../shared/references.bib}

\usepackage{amsmath,amssymb}
% --- Graphics / TikZ ---
\usepackage{tikz}
\usetikzlibrary{arrows.meta,calc,decorations.pathreplacing}
% shared/macros.tex
\providecommand{\Afive}{A_{5}}
\providecommand{\golden}{\varphi} % optional standard name for \varphi
\providecommand{\phig}{\varphi}
\providecommand{\dd}{\mathrm{d}}
\providecommand{\ii}{\mathrm{i}}
\providecommand{\ee}{\mathrm{e}}
\providecommand{\Tr}{\mathrm{Tr}}
\providecommand{\diag}{\mathrm{diag}}
\providecommand{\SU}{\mathrm{SU}}
\providecommand{\U}{\mathrm{U}}


% ============================================================
% GPP_DRAFT_SUPPRESS_WARNINGS
% Draft-mode suppression of undefined reference warnings.
% Toggle: set \GPPdrafttrue (on) / \GPPdraftfalse (off)
% ============================================================
\newif\ifGPPdraft
\GPPdrafttrue

\makeatletter
\ifGPPdraft
  % Suppress only warning emissions (keeps errors, overfull boxes, etc.)
  \def\@latex@warning@no@line#1{}
  \def\@latex@warning#1{}
\fi
\makeatother



\title{Paper III: Anchored Integer-Lattice Fits for Particle Masses}
\author{[Author Name]}
\date{\today}

\begin{document}
\maketitle

\begin{abstract}
We formalize and report the results of an anchored integer-lattice model for the Standard Model mass spectrum.
Masses are represented in logarithmic coordinates using the golden ratio $\golden=(1+\sqrt5)/2$ as a base.
The model assigns each particle an integer triple $(a,b,c)\in\mathbb{Z}^3$ and a derived index
$q=8a+15b+24c$. Predicted masses obey $m=m_e\,\golden^{q/4}$ after anchoring the electron as a reference state
($q_e=0$). We document the search protocol, error metric, multiplicity accounting, and present best-fit solutions
in a reproducible format.
\end{abstract}

\section{Model definition}
\label{sec:paperIII-model}

\subsection{Logarithmic mass coordinate}
Let $\golden=(1+\sqrt5)/2$. For any mass $m>0$ define the logarithmic coordinate
\begin{equation}
\ell(m)\;=\;\log_{\golden}\!\left(\frac{m}{m_e}\right),
\end{equation}
where $m_e$ is the electron mass (in consistent units).

\subsection{Integer-lattice hypothesis}
We posit that for a chosen set of particles, the coordinates $\ell(m)$ lie near a discrete lattice generated by three
integer parameters $(a,b,c)$ through a linear form
\begin{equation}
q \;=\; 8a + 15b + 24c,
\label{eq:qdef}
\end{equation}
with predicted mass
\begin{equation}
m_{\mathrm{pred}}(a,b,c)\;=\;m_e\,\golden^{q/4}.
\label{eq:mpred}
\end{equation}
The coefficients $\{8,15,24\}$ are treated here as fixed generators (motivated elsewhere by $\Afive$-linked constructions);
Paper III focuses on the \emph{anchored fitting procedure} and empirical outputs rather than a full derivation of these
coefficients.

\subsection{Anchoring convention}
\label{sec:anchoring}
Direct scans of \eqref{eq:mpred} exhibit a trivial degeneracy under global shifts of $(a,b,c)$ that preserve relative
differences in $q$. To remove this, we anchor the model by assigning the electron a reference triple $(a_e,b_e,c_e)$ and
defining the anchored index
\begin{equation}
q_{\mathrm{anch}} \;=\; (8a+15b+24c) - (8a_e+15b_e+24c_e),
\end{equation}
so that the electron satisfies $q_{\mathrm{anch}}=0$ and $m_{\mathrm{pred}}=m_e$ by construction. All other particles are
fit using $q_{\mathrm{anch}}$.

\section{Fitting protocol}
\label{sec:paperIII-protocol}

\subsection{Search region}
For each particle we scan integer triples $(a,b,c)$ within bounded boxes
\begin{equation}
a\in[a_{\min},a_{\max}],\quad b\in[b_{\min},b_{\max}],\quad c\in[c_{\min},c_{\max}],
\end{equation}
with ranges chosen to ensure coverage of the relevant decade structure while remaining computationally tractable.

\subsection{Error metric}
We evaluate the relative logarithmic error
\begin{equation}
\epsilon \;=\;\frac{\log(m_{\mathrm{pred}}/m_{\mathrm{exp}})}{\log(m_{\mathrm{exp}}/m_e)}.
\label{eq:eps}
\end{equation}
This metric penalizes mismatch proportionally in log-space and is appropriate when the hypothesis concerns structure in
$\ell(m)$ rather than linear mass differences. For each particle we report the best-fit triple minimizing $|\epsilon|$.

\subsection{Multiplicity accounting}
Because multiple integer triples can map to the same $q$ (and hence the same prediction), we deduplicate solutions by
the value of $q$ when reporting multiplicities. A tolerance threshold $|\epsilon|\le\epsilon_{\max}$ defines the accepted
solution set. We report:
(i) the number of distinct $q$ values within tolerance, and
(ii) the smallest achieved $|\epsilon|$ in the scan region.

\section{Results}
\label{sec:paperIII-results-main}

This section is generated from the same dataset/scripts used to produce the repository tables, and is included verbatim
for reproducibility.

% DUPLICATE REMOVED: % Auto-generated results artifact (mass-lattice summary).
% This file is intended to be INPUT into Paper I.
% Do NOT place \documentclass, \begin{document}, or any self-referential \input here.
% Do NOT define sectioning commands or labels here.

\begin{center}
\textbf{Auto-generated mass-lattice results (placeholder)}
\end{center}

\noindent
The audited results artifact was not regenerated in this repo copy. To restore the full numerical
tables, re-run the artifact generator in the Geometric_Particle_Physics workflow and overwrite
\texttt{shared/paperIII_results.tex} with the generated output.

% End of artifact.


\section{Interpretation and immediate implications}
\label{sec:paperIII-interpretation}

Three points are worth emphasizing.

\paragraph{Anchoring removes an unphysical degree of freedom.}
Without anchoring, any global shift in the integer parameters can mimic ``new solutions'' without changing the underlying
structure. Anchoring forces the electron to be the reference state and makes the reported triples comparable across runs.

\paragraph{Uniqueness must be discussed in terms of $q$.}
Because $m_{\mathrm{pred}}$ depends only on $q$, the physically relevant degeneracy class is indexed by $q$, not by the raw
triples. Future statistical tests should therefore treat different triples with identical $q$ as a single prediction.

\paragraph{What this paper does and does not claim.}
Paper III establishes a controlled definition of the lattice fit problem and reports reproducible outputs. It does not,
by itself, demonstrate a dynamical origin of \eqref{eq:qdef} or prove that the pattern is statistically unexpected under all
reasonable null hypotheses. Those questions are addressed in later papers via (i) broader datasets, (ii) robustness under
scheme/running updates, and (iii) explicit falsifiability criteria.

\section{Reproducibility checklist}
\label{sec:paperIII-repro}

A submission-ready version of this paper requires:
\begin{enumerate}
\item A fixed, versioned input table of experimental masses and uncertainties (with sources).
\item A deterministic script that regenerates the best-fit table and multiplicity counts.
\item A pre-registered null ensemble for significance testing (e.g., randomized log-masses with preserved ordering).
\item A sensitivity analysis over scan bounds and tolerance $\epsilon_{\max}$.
\end{enumerate}

\end{document}
