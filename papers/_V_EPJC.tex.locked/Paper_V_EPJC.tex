\documentclass[12pt,a4paper]{article}

% =========================
% Packages (EPJC-safe set)
% =========================
\usepackage[T1]{fontenc}
\usepackage[utf8]{inputenc}
\usepackage{lmodern}
\usepackage{geometry}
\usepackage{amsmath,amssymb,amsfonts}
\usepackage{graphicx}
\usepackage{hyperref}
\usepackage{booktabs}
\usepackage{enumitem}
\usepackage{cite}

\geometry{margin=1in}

% =========================
% Metadata
% =========================
\title{\textbf{Geometric Regularity and Predictive Structure\\
in Anchored Modular Flavor Models}}

\author{Marvin L. Gentry\\
\small Independent Researcher\\
\small ORCID: 0009-0006-4550-2663\\
\small \texttt{drmlgentry@protonmail.com}}

\date{\today}

% =========================
\begin{document}
\maketitle

% =========================
\begin{abstract}
We present a structural synthesis of an anchored modular flavor framework developed across preceding works, emphasizing its predictive content and phenomenological constraints. Rather than introducing new fits or scans, this paper isolates the geometric principles underlying the observed fermion mass regularities and clarifies which features are robust, which are conventional, and which admit falsifiable extensions. The framework is shown to naturally encode hierarchical mass spectra through discrete geometric anchoring while remaining compatible with known limitations arising from scale dependence, scheme choices, and parameter multiplicity. We outline explicit directions in which the construction may be extended toward testable predictions beyond the Standard Model.
\end{abstract}

% =========================
\section{Introduction}

The fermion mass spectrum of the Standard Model exhibits striking hierarchical structure without an established organizing principle. Numerous approaches invoke flavor symmetries, texture ans\"atze, or statistical arguments, yet few provide a unifying geometric interpretation that remains transparent under perturbations and scale changes.

In a sequence of prior papers, we introduced an anchored modular framework in which fermion masses are associated with discrete integer data subject to geometric constraints. The present work does not extend that construction numerically. Instead, it clarifies the conceptual status of the framework, consolidates its interpretive content, and articulates the extent to which its regularities should be regarded as structural rather than coincidental.

This paper is intended as a closure of the interpretive phase of the program and as a bridge toward predictive applications.

% =========================
\section{Structural Summary of the Anchored Framework}

The central object of the framework is an assignment
\begin{equation}
m_f \;\longleftrightarrow\; (a_f,b_f,c_f),
\end{equation}
where each fermion mass is associated with an integer triple relative to a fixed reference anchor. Physical masses enter only through logarithmic ratios, ensuring that overall scale conventions do not affect the discrete structure.

Two features are essential:

\begin{itemize}[leftmargin=2em]
\item \textbf{Anchoring:} A single reference triple fixes the origin of the integer lattice. Different anchors correspond to equivalent descriptions related by integer translations.
\item \textbf{Discrete Geometry:} Regularities emerge not from fine tuning but from the density and organization of integer points satisfying approximate constraints.
\end{itemize}

Importantly, the framework does not assume uniqueness of solutions. Multiplicity is treated as a diagnostic rather than a defect.

% =========================
\section{Statistical Interpretation}

The appearance of close fits within bounded integer regions should not be interpreted as evidence of exact relations. Instead, the framework supports a statistical reading:

\begin{itemize}[leftmargin=2em]
\item Regularity indicates structure in the mapping between continuous masses and discrete data.
\item Multiplicity reflects degeneracy of representation rather than physical redundancy.
\item Precision thresholds must be interpreted relative to scan volume and density.
\end{itemize}

Under this view, the framework functions as a coarse-grained encoding of mass hierarchies rather than a fine-tuned reconstruction.

% =========================
\section{Limitations and Robustness}

Several limitations are intrinsic and should be stated explicitly:

\begin{enumerate}[leftmargin=2em]
\item \textbf{Running Effects:} Fermion masses depend on renormalization scale and scheme. The framework assumes fixed inputs and does not yet incorporate full running.
\item \textbf{Non-Uniqueness:} No claim of unique integer assignments is made or implied.
\item \textbf{Model Incompleteness:} The construction is kinematic rather than dynamical; no underlying force law is specified.
\end{enumerate}

These limitations do not undermine the observed structural regularities but delimit their interpretive scope.

% =========================
\section{Position Within the Broader Program}

This work completes the interpretive phase of a broader research program aimed at identifying geometric organization in flavor physics. Earlier papers focused on construction and empirical alignment; the present paper isolates meaning and constraints.

Future extensions may include:

\begin{itemize}[leftmargin=2em]
\item incorporation of renormalization group flow,
\item extension to neutrino sectors,
\item coupling to modular or discrete symmetry dynamics,
\item exploration of CP-violating observables.
\end{itemize}

These directions move beyond descriptive regularity toward predictive structure.

% =========================
\section{Conclusion}

We have consolidated the conceptual content of an anchored modular flavor framework, emphasizing its geometric interpretation and statistical reading. While the framework does not claim exact mass relations, it reveals nontrivial organization in the fermion spectrum that warrants further investigation. By clarifying what is structural, what is conventional, and what remains open, this paper establishes a stable foundation for future theoretical development.

% =========================
\bibliographystyle{unsrt}
\begin{thebibliography}{9}

\bibitem{PDG}
Particle Data Group,
\textit{Review of Particle Physics},
Prog.\ Theor.\ Exp.\ Phys.\ \textbf{2024}, 083C01 (2024).

\bibitem{FlavorReview}
G.~Altarelli and F.~Feruglio,
\textit{Discrete Flavor Symmetries and Models of Neutrino Mixing},
Rev.\ Mod.\ Phys.\ \textbf{82}, 2701 (2010).

\end{thebibliography}

\end{document}
