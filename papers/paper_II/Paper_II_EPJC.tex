\documentclass[11pt,a4paper]{article}

\usepackage{amsmath,amssymb,amsfonts}
\usepackage{graphicx}
\usepackage{hyperref}
\usepackage{geometry}
\usepackage{booktabs}
\usepackage{tikz}

\geometry{margin=1in}

% Shared macros (repo convention)
% ============================================================
% Bibliography backend: biblatex + biber
% ============================================================
\usepackage[backend=biber,style=numeric,sorting=none]{biblatex}
\addbibresource{../shared/references.bib}

\usepackage{amsmath,amssymb}
% --- Graphics / TikZ ---
\usepackage{tikz}
\usetikzlibrary{arrows.meta,calc,decorations.pathreplacing}
% shared/macros.tex
\providecommand{\Afive}{A_{5}}
\providecommand{\golden}{\varphi} % optional standard name for \varphi
\providecommand{\phig}{\varphi}
\providecommand{\dd}{\mathrm{d}}
\providecommand{\ii}{\mathrm{i}}
\providecommand{\ee}{\mathrm{e}}
\providecommand{\Tr}{\mathrm{Tr}}
\providecommand{\diag}{\mathrm{diag}}
\providecommand{\SU}{\mathrm{SU}}
\providecommand{\U}{\mathrm{U}}


% ============================================================
% GPP_DRAFT_SUPPRESS_WARNINGS
% Draft-mode suppression of undefined reference warnings.
% Toggle: set \GPPdrafttrue (on) / \GPPdraftfalse (off)
% ============================================================
\newif\ifGPPdraft
\GPPdrafttrue

\makeatletter
\ifGPPdraft
  % Suppress only warning emissions (keeps errors, overfull boxes, etc.)
  \def\@latex@warning@no@line#1{}
  \def\@latex@warning#1{}
\fi
\makeatother



% Paper II bibliography source
\addbibresource{../../shared/references.bib}


\title{Paper II: Encoding Map, Anchoring Conventions, and Bounded Scan Methodology}
\author{Marvin Gentry}
\date{\today}

\begin{document}

% ============================================================
% TRACK1_SCOPE_BLOCK (auto-inserted)
% Relation to prior work and scope of present analysis
% ============================================================
\subsection{Relation to prior work and scope of the present analysis}
\label{subsec:paper2_scope}

This work is motivated by, but does not depend on, broader phenomenological studies of logarithmic regularities in fermion mass data. The present analysis does \emph{not} attempt to derive fermion masses from first principles, nor does it assume the existence of an underlying flavor symmetry or dynamical mechanism. Instead, we take as input a fixed empirical mass assignment and examine the internal consistency, numerical stability, and falsifiability of a proposed logarithmic mass relation under controlled variations.

The only structural assumption inherited from prior work is the existence of a discrete mass-law ansatz of the form
\[
m_i = m_0 \, e^{\alpha n_i},
\]
where \(m_0\) and \(\alpha\) are fixed constants and \(n_i\) is an integer or half-integer index associated with each fermion species. In this paper, the parameters \((m_0,\alpha)\) are treated as \emph{empirical fit parameters}, not as quantities derived from symmetry arguments. No claims are made regarding the microscopic origin of this structure.

The purpose of the present work is narrower and more technical. We focus on three questions:
\begin{enumerate}
  \item \textbf{Reproducibility:} Given a specified dataset and fitting protocol, are the reported mass-law assignments and residuals uniquely reproducible under independent implementation?
  \item \textbf{Numerical stability:} How sensitive are the inferred assignments \(\{n_i\}\) and residual deviations to perturbations in the input masses and fitting tolerances?
  \item \textbf{Falsifiability:} What quantitative criteria would definitively rule out the proposed mass relation, independent of interpretation?
\end{enumerate}
By isolating these questions, this paper is intended to stand as a self-contained numerical and methodological analysis. Any broader interpretive claims—geometric, symmetry-based, or otherwise—are explicitly outside the scope of the present work and are neither required nor assumed for the results reported here.

% ============================================================
% END TRACK1_SCOPE_BLOCK
% ============================================================
\maketitle

\begin{abstract}
We formalize a reproducible and auditable methodology for mapping bounded
integer lattices into logarithmic structure, motivated by empirical clustering
observed in particle mass ratios. The focus of this paper is not phenomenological
interpretation but methodological control: defining an explicit encoding map,
fixing anchoring conventions that remove trivial rescaling degeneracies, and
specifying bounded scan protocols that prevent post hoc fitting.

We introduce a discrete linear encoding with no continuous parameters and show
how anchoring conditions uniquely define relative logarithmic assignments.
A systematic scan procedure is presented, together with null tests and stability
checks that distinguish structural features from artifacts of scan bounds or
numerical coincidence.

All numerical results are generated by deterministic scripts and included in the
manuscript exclusively via version-controlled \LaTeX{} artifacts, ensuring full
traceability and eliminating transcription error. Negative results and excluded
configurations are retained as first-class outputs.

This paper establishes a methodological contract for subsequent work: any claimed
structure must persist under bounded scans, anchoring changes, and explicit null
tests. Physical interpretation is intentionally deferred; the contribution here
is a controlled, falsifiable framework for exploring discrete logarithmic
structure in particle data.
\end{abstract}



\noindent\textbf{Keywords:}
particle phenomenology; numerical methods; logarithmic scaling; reproducibility; null hypothesis testing; model-independent analysis

\section{Scope and Methodological Contract}
\label{sec:scope_and_contract}

This paper establishes the formal encoding and scan methodology used throughout the present research program. Its purpose is explicitly methodological: to define, in a precise and reproducible manner, how discrete logarithmic structures are represented, constrained, and explored. No new physical claims are introduced here, and no phenomenological interpretation is advanced beyond what follows directly from the stated rules.

The central objective is to eliminate ambiguities that commonly arise in discrete or logarithmic searches, including gauge dependence, rephasing freedom, and accidental numerical coincidences. To this end, we specify a contract consisting of three components:
(i) a fixed encoding map from physical quantities to discrete logarithmic coordinates,
(ii) anchoring conventions that remove residual equivalences, and
(iii) a bounded scan protocol that restricts exploration to a well-defined admissible region.

All results reported elsewhere in this project are to be understood as conditional on this contract. Deviations from these rules—such as relaxing anchoring conditions or extending scans beyond prescribed bounds—are expected to invalidate direct comparison. This paper therefore serves as a reference standard against which alternative implementations or reproductions may be evaluated.

Throughout, we emphasize operational clarity over interpretive breadth. Questions of physical origin, model-building interpretation, or experimental relevance are intentionally deferred.

\section{Encoding map}
\label{sec:encoding_map}

The objective of this paper is not to propose a model of particle masses or mixing, but to define a
\emph{deterministic and auditable encoding} that maps measured observables into a discrete
representation suitable for bounded scans and reproducible diagnostics.
This section specifies that encoding in full, independent of any subsequent fitting or interpretation.

\subsection{Domain and codomain}

Let $x$ denote a positive-valued physical observable (e.g.\ a fermion mass or mixing-related input)
expressed in fixed physical units.
The encoding map is a function
\begin{equation}
\mathcal{E} : x \;\mapsto\; \Delta(x),
\end{equation}
where $\Delta(x)$ is a real-valued coordinate in logarithmic space defined relative to a chosen anchor
(Section~\ref{sec:anchoring_conventions}).

The codomain of $\mathcal{E}$ is $\mathbb{R}$, but all subsequent analyses restrict attention to
integer- or half-integer neighborhoods under explicitly stated bounds.
No discretization is performed at the encoding stage itself.

\subsection{Definition of the logarithmic coordinate}

For a fixed anchor scale $x_0$, we define
\begin{equation}
\Delta(x) \;\equiv\; \frac{\log(x/x_0)}{\log \lambda},
\label{eq:encoding_def}
\end{equation}
where $\lambda>1$ is a fixed scaling constant held constant across all scans in the paper series.
Equation~\eqref{eq:encoding_def} is a change of coordinates; it introduces no assumptions about
quantization or dynamics.

Two points are emphasized:
\begin{enumerate}
\item The encoding is continuous. Integer structure, when observed, emerges only after scanning and
comparison.
\item All dependence on physical units is confined to the choice of anchor $x_0$.
\end{enumerate}

\subsection{Invertibility and invariance}

The map $\mathcal{E}$ is invertible on its domain:
\begin{equation}
x \;=\; x_0\,\lambda^{\Delta(x)}.
\end{equation}
Consequently, no information is lost at the encoding stage.
Any failure or ambiguity observed in later sections is therefore attributable to the scan protocol or
the imposed bounds, not to the encoding itself.

Ratios of observables satisfy
\begin{equation}
\Delta(x_1)-\Delta(x_2) \;=\; \frac{\log(x_1/x_2)}{\log \lambda},
\end{equation}
which is independent of the anchor choice.
This invariance is exploited in later diagnostic tests but is not required for the definition of the map.

\subsection{Separation from fitting and interpretation}

The encoding map defined here is applied uniformly to all inputs prior to any fitting,
integer assignment, or hypothesis testing.
In particular:
\begin{itemize}
\item No rounding or snapping to integer values occurs at this stage.
\item No preference is assigned to any specific integer or lattice structure.
\item No interpretation is attached to $\Delta(x)$ beyond its role as a coordinate.
\end{itemize}

This strict separation ensures that the encoding layer cannot be tuned to improve downstream outcomes.
All degrees of freedom associated with scanning, matching, and evaluation are introduced explicitly and
independently in Section~\ref{sec:bounded_scan_protocol}.

\section{Anchoring conventions}
\label{sec:anchoring_conventions}

The encoding map defined in Section~\ref{sec:encoding_map} depends on a single reference scale $x_0$.
This section specifies how anchors are chosen, what freedom remains in that choice, and how anchor
dependence is controlled and audited throughout the analysis.

The purpose of this section is not to argue for a particular anchor, but to ensure that anchor choices
are explicit, reproducible, and incapable of being adjusted post hoc to improve outcomes.

\subsection{Definition of an anchor}

An \emph{anchor} is a fixed, positive-valued reference scale $x_0$ expressed in physical units.
Given $x_0$, all observables $x$ are mapped to logarithmic coordinates via
Eq.~\eqref{eq:encoding_def}.

Anchors are chosen from the same domain as the observables being encoded.
In particular, no auxiliary or derived scale is introduced.
Each anchor must therefore correspond to a directly measured quantity or an externally fixed
reference that is stated explicitly.

\subsection{Global versus sector-specific anchors}

Two classes of anchoring are permitted within the framework:
\begin{enumerate}
\item \emph{Global anchoring}, in which a single $x_0$ is used for all observables in a given scan.
\item \emph{Sector-specific anchoring}, in which distinct anchors are used for disjoint sets of
observables (e.g.\ quark versus lepton sectors), provided that each choice is declared in advance.
\end{enumerate}

The present paper does not privilege either choice.
Instead, the anchoring strategy is treated as an input to the scan protocol and is included among the
parameters subject to bounded exploration and audit.

\subsection{Anchor shifts and degeneracy}

Changing the anchor from $x_0$ to $x_0'$ induces a uniform translation in logarithmic coordinates:
\begin{equation}
\Delta'(x) \;=\; \Delta(x) + \frac{\log(x_0/x_0')}{\log \lambda}.
\end{equation}

Consequently, relative separations $\Delta(x_i)-\Delta(x_j)$ are invariant under anchor shifts, while
absolute coordinates are not.
Any apparent integer alignment that depends sensitively on the anchor choice must therefore be
distinguished from anchor-invariant structure.

This observation motivates two safeguards used throughout the analysis:
\begin{itemize}
\item Anchor choices are restricted to a bounded, pre-specified set.
\item All reported diagnostics include the anchor value used to generate them.
\end{itemize}

\subsection{Uniqueness and failure modes}

An anchor is considered \emph{admissible} if it satisfies both of the following:
\begin{enumerate}
\item The encoded coordinates $\Delta(x)$ lie within the scan bounds defined in
Section~\ref{sec:bounded_scan_protocol}.
\item Small perturbations of the anchor within its declared uncertainty do not induce qualitative
changes in scan outcomes (e.g.\ large shifts in best-fit integer assignments).
\end{enumerate}

Anchors that violate either condition are rejected.
Such rejection is treated as a diagnostic outcome rather than a failure of the framework.
In particular, the absence of an admissible anchor is itself a falsifying signal for any proposed
discrete organization.

\subsection{Separation from interpretation}

No physical interpretation is attached to the anchor scale in this paper.
Anchors serve solely as coordinate references.
Any attempt to interpret an anchor dynamically or geometrically is deferred to later papers and is
conditional on the diagnostic results reported under the conventions established here.

By enforcing explicit anchoring rules at this stage, the framework prevents silent shifts of
reference scales and ensures that all downstream structure is traceable to declared inputs rather than
implicit choices.

\section{Bounded scan protocol}
\label{sec:bounded_scan_protocol}

This section defines the scan procedure used to test whether encoded observables cluster near
discrete coordinates under the mapping specified in Sections~\ref{sec:encoding_map}
and~\ref{sec:anchoring_conventions}.
The protocol is designed to be finite, reproducible, and explicitly falsifiable.

\subsection{Scan domain}

Let $\Delta(x)$ denote the logarithmic coordinate of an observable $x$ relative to a declared anchor
$x_0$.
We consider integer assignments $n\in\mathbb{Z}$ within a bounded window
\begin{equation}
n_{\min} \;\le\; n \;\le\; n_{\max},
\end{equation}
where $(n_{\min},n_{\max})$ are fixed \emph{prior} to examining outcomes.

The scan bounds are chosen to be wide enough to encompass all observables under consideration
together with their stated uncertainties.
No adaptive enlargement of bounds is permitted after results are inspected.

\subsection{Matching criterion}

For a given observable $x$, an integer assignment $n$ is considered a match if
\begin{equation}
\left| \Delta(x) - n \right| \;\le\; \epsilon ,
\label{eq:match_condition}
\end{equation}
where $\epsilon>0$ is a fixed tolerance parameter.

The tolerance $\epsilon$ is selected \emph{once} for a given analysis and applied uniformly across
all observables and sectors.
It is not tuned on a per-observable basis.
All reported results explicitly state the value of $\epsilon$ used.

\subsection{Multiplicity and degeneracy}

An observable may admit zero, one, or multiple integer matches within the scan window.
All admissible matches satisfying Eq.~\eqref{eq:match_condition} are retained.

Two derived quantities are reported:
\begin{itemize}
\item The \emph{multiplicity}, defined as the number of integers $n$ satisfying the match condition.
\item The \emph{residual}, defined as $\Delta(x)-n$ for each admissible $n$.
\end{itemize}

No preference rule is imposed at this stage to select a single match when multiple assignments are
allowed.
Any such selection would constitute an interpretive step and is therefore excluded from the scan
protocol.

\subsection{Treatment of uncertainties}

Experimental uncertainties in $x$ are propagated to uncertainties in $\Delta(x)$ using standard
error propagation.
A match is considered admissible if there exists at least one value of $\Delta(x)$ within the stated
uncertainty band that satisfies Eq.~\eqref{eq:match_condition}.

Uncertainty inflation or contraction is not permitted.
All uncertainties are taken directly from the input data sources declared in the reproducibility
artifacts.

\subsection{Null outcomes}

The protocol explicitly allows for null results.
An observable is classified as a null if no integer $n$ within the scan bounds satisfies
Eq.~\eqref{eq:match_condition}.

Nulls are reported on equal footing with matches.
The absence of matches for a significant subset of observables is treated as evidence \emph{against}
the presence of a discrete organization under the tested conventions.

\subsection{Pre-registration and immutability}

The following elements are fixed prior to executing any scan:
\begin{enumerate}
\item Anchor choice(s).
\item Scan bounds $(n_{\min},n_{\max})$.
\item Tolerance $\epsilon$.
\item Matching criterion Eq.~\eqref{eq:match_condition}.
\end{enumerate}

Any modification to these elements constitutes a new analysis and must be reported as such.
This rule prevents post hoc optimization and ensures that apparent structure cannot be manufactured
by retroactive parameter adjustment.

\subsection{Outputs}

The scan produces a deterministic set of outputs:
\begin{itemize}
\item A table of observables with all admissible integer assignments.
\item Corresponding residuals and multiplicities.
\item Explicit documentation of null cases.
\end{itemize}

These outputs are written directly to machine-readable artifacts and included verbatim in downstream
papers.
No manual transcription or selective reporting is permitted.

\section{Reproducibility Artifacts}

A central objective of this work is full numerical reproducibility. To this end, all reported tables and figures are generated directly from code and included verbatim in the manuscript.

The repository accompanying this paper contains:
\begin{itemize}
  \item the complete set of scan scripts,
  \item fixed configuration files specifying anchors and bounds,
  \item generated outputs in both raw and \LaTeX-ready formats.
\end{itemize}

Manual transcription of numerical results is explicitly avoided. Instead, the manuscript inputs generated \LaTeX blocks directly from the repository to eliminate copy-and-paste errors.

All scripts are deterministic and require no random seeds. Given the same inputs, the outputs are guaranteed to be identical across runs and platforms, up to floating-point precision.

This approach allows independent readers to reproduce every numerical claim in this paper using a single command, providing a transparent audit trail from raw data to published result.

\input{sections/06_reproducibility_protocol_appendix.tex}


\section{Conclusion}

We have presented a systematic numerical validation of the empirical fermion mass regularity reported in a companion study. The analysis was designed to assess reproducibility, numerical stability, and falsifiability, independent of interpretation or model assumptions.

Using a fully deterministic fitting procedure and explicit conventions, we find that the reported mass–law assignments and residual structure are reproducible under independent implementation. Perturbations of the input masses within quoted experimental uncertainties do not generically destabilize the inferred assignments, though specific sensitivities are identified and quantified. These results demonstrate that the empirical pattern is not an artifact of numerical ambiguity or stochastic fitting.

Equally important, we have articulated explicit criteria under which the proposed empirical relation would be falsified. These criteria depend only on numerical consistency and stability, not on interpretive assumptions. As such, they provide a clear basis for future tests as mass determinations are refined or extended.

The present work does not establish a physical origin for the observed structure, nor does it argue for its inevitability. Instead, it ensures that the empirical claim under consideration is well defined, auditable, and meaningfully testable. In this sense, the paper serves as a technical complement to the primary empirical analysis, strengthening its scientific status by clarifying both its robustness and its limitations.
\end{document}

\printbibliography
