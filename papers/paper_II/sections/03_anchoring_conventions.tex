\section{Anchoring conventions}
\label{sec:anchoring_conventions}

The encoding map defined in Section~\ref{sec:encoding_map} depends on a single reference scale $x_0$.
This section specifies how anchors are chosen, what freedom remains in that choice, and how anchor
dependence is controlled and audited throughout the analysis.

The purpose of this section is not to argue for a particular anchor, but to ensure that anchor choices
are explicit, reproducible, and incapable of being adjusted post hoc to improve outcomes.

\subsection{Definition of an anchor}

An \emph{anchor} is a fixed, positive-valued reference scale $x_0$ expressed in physical units.
Given $x_0$, all observables $x$ are mapped to logarithmic coordinates via
Eq.~\eqref{eq:encoding_def}.

Anchors are chosen from the same domain as the observables being encoded.
In particular, no auxiliary or derived scale is introduced.
Each anchor must therefore correspond to a directly measured quantity or an externally fixed
reference that is stated explicitly.

\subsection{Global versus sector-specific anchors}

Two classes of anchoring are permitted within the framework:
\begin{enumerate}
\item \emph{Global anchoring}, in which a single $x_0$ is used for all observables in a given scan.
\item \emph{Sector-specific anchoring}, in which distinct anchors are used for disjoint sets of
observables (e.g.\ quark versus lepton sectors), provided that each choice is declared in advance.
\end{enumerate}

The present paper does not privilege either choice.
Instead, the anchoring strategy is treated as an input to the scan protocol and is included among the
parameters subject to bounded exploration and audit.

\subsection{Anchor shifts and degeneracy}

Changing the anchor from $x_0$ to $x_0'$ induces a uniform translation in logarithmic coordinates:
\begin{equation}
\Delta'(x) \;=\; \Delta(x) + \frac{\log(x_0/x_0')}{\log \lambda}.
\end{equation}

Consequently, relative separations $\Delta(x_i)-\Delta(x_j)$ are invariant under anchor shifts, while
absolute coordinates are not.
Any apparent integer alignment that depends sensitively on the anchor choice must therefore be
distinguished from anchor-invariant structure.

This observation motivates two safeguards used throughout the analysis:
\begin{itemize}
\item Anchor choices are restricted to a bounded, pre-specified set.
\item All reported diagnostics include the anchor value used to generate them.
\end{itemize}

\subsection{Uniqueness and failure modes}

An anchor is considered \emph{admissible} if it satisfies both of the following:
\begin{enumerate}
\item The encoded coordinates $\Delta(x)$ lie within the scan bounds defined in
Section~\ref{sec:bounded_scan_protocol}.
\item Small perturbations of the anchor within its declared uncertainty do not induce qualitative
changes in scan outcomes (e.g.\ large shifts in best-fit integer assignments).
\end{enumerate}

Anchors that violate either condition are rejected.
Such rejection is treated as a diagnostic outcome rather than a failure of the framework.
In particular, the absence of an admissible anchor is itself a falsifying signal for any proposed
discrete organization.

\subsection{Separation from interpretation}

No physical interpretation is attached to the anchor scale in this paper.
Anchors serve solely as coordinate references.
Any attempt to interpret an anchor dynamically or geometrically is deferred to later papers and is
conditional on the diagnostic results reported under the conventions established here.

By enforcing explicit anchoring rules at this stage, the framework prevents silent shifts of
reference scales and ensures that all downstream structure is traceable to declared inputs rather than
implicit choices.
