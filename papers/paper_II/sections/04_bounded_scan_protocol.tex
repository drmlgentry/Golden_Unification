\section{Bounded scan protocol}
\label{sec:bounded_scan_protocol}

This section defines the scan procedure used to test whether encoded observables cluster near
discrete coordinates under the mapping specified in Sections~\ref{sec:encoding_map}
and~\ref{sec:anchoring_conventions}.
The protocol is designed to be finite, reproducible, and explicitly falsifiable.

\subsection{Scan domain}

Let $\Delta(x)$ denote the logarithmic coordinate of an observable $x$ relative to a declared anchor
$x_0$.
We consider integer assignments $n\in\mathbb{Z}$ within a bounded window
\begin{equation}
n_{\min} \;\le\; n \;\le\; n_{\max},
\end{equation}
where $(n_{\min},n_{\max})$ are fixed \emph{prior} to examining outcomes.

The scan bounds are chosen to be wide enough to encompass all observables under consideration
together with their stated uncertainties.
No adaptive enlargement of bounds is permitted after results are inspected.

\subsection{Matching criterion}

For a given observable $x$, an integer assignment $n$ is considered a match if
\begin{equation}
\left| \Delta(x) - n \right| \;\le\; \epsilon ,
\label{eq:match_condition}
\end{equation}
where $\epsilon>0$ is a fixed tolerance parameter.

The tolerance $\epsilon$ is selected \emph{once} for a given analysis and applied uniformly across
all observables and sectors.
It is not tuned on a per-observable basis.
All reported results explicitly state the value of $\epsilon$ used.

\subsection{Multiplicity and degeneracy}

An observable may admit zero, one, or multiple integer matches within the scan window.
All admissible matches satisfying Eq.~\eqref{eq:match_condition} are retained.

Two derived quantities are reported:
\begin{itemize}
\item The \emph{multiplicity}, defined as the number of integers $n$ satisfying the match condition.
\item The \emph{residual}, defined as $\Delta(x)-n$ for each admissible $n$.
\end{itemize}

No preference rule is imposed at this stage to select a single match when multiple assignments are
allowed.
Any such selection would constitute an interpretive step and is therefore excluded from the scan
protocol.

\subsection{Treatment of uncertainties}

Experimental uncertainties in $x$ are propagated to uncertainties in $\Delta(x)$ using standard
error propagation.
A match is considered admissible if there exists at least one value of $\Delta(x)$ within the stated
uncertainty band that satisfies Eq.~\eqref{eq:match_condition}.

Uncertainty inflation or contraction is not permitted.
All uncertainties are taken directly from the input data sources declared in the reproducibility
artifacts.

\subsection{Null outcomes}

The protocol explicitly allows for null results.
An observable is classified as a null if no integer $n$ within the scan bounds satisfies
Eq.~\eqref{eq:match_condition}.

Nulls are reported on equal footing with matches.
The absence of matches for a significant subset of observables is treated as evidence \emph{against}
the presence of a discrete organization under the tested conventions.

\subsection{Pre-registration and immutability}

The following elements are fixed prior to executing any scan:
\begin{enumerate}
\item Anchor choice(s).
\item Scan bounds $(n_{\min},n_{\max})$.
\item Tolerance $\epsilon$.
\item Matching criterion Eq.~\eqref{eq:match_condition}.
\end{enumerate}

Any modification to these elements constitutes a new analysis and must be reported as such.
This rule prevents post hoc optimization and ensures that apparent structure cannot be manufactured
by retroactive parameter adjustment.

\subsection{Outputs}

The scan produces a deterministic set of outputs:
\begin{itemize}
\item A table of observables with all admissible integer assignments.
\item Corresponding residuals and multiplicities.
\item Explicit documentation of null cases.
\end{itemize}

These outputs are written directly to machine-readable artifacts and included verbatim in downstream
papers.
No manual transcription or selective reporting is permitted.
