\section{Scope and Methodological Contract}
\label{sec:scope_and_contract}

This paper establishes the formal encoding and scan methodology used throughout the present research program. Its purpose is explicitly methodological: to define, in a precise and reproducible manner, how discrete logarithmic structures are represented, constrained, and explored. No new physical claims are introduced here, and no phenomenological interpretation is advanced beyond what follows directly from the stated rules.

The central objective is to eliminate ambiguities that commonly arise in discrete or logarithmic searches, including gauge dependence, rephasing freedom, and accidental numerical coincidences. To this end, we specify a contract consisting of three components:
(i) a fixed encoding map from physical quantities to discrete logarithmic coordinates,
(ii) anchoring conventions that remove residual equivalences, and
(iii) a bounded scan protocol that restricts exploration to a well-defined admissible region.

All results reported elsewhere in this project are to be understood as conditional on this contract. Deviations from these rules—such as relaxing anchoring conditions or extending scans beyond prescribed bounds—are expected to invalidate direct comparison. This paper therefore serves as a reference standard against which alternative implementations or reproductions may be evaluated.

Throughout, we emphasize operational clarity over interpretive breadth. Questions of physical origin, model-building interpretation, or experimental relevance are intentionally deferred.
