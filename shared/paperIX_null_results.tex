% ============================================================
% Paper IX — Null hypotheses and statistical protocol (RESULTS)
% Auto-authored block for % ============================================================
% Paper IX — Null hypotheses and statistical protocol (RESULTS)
% Auto-authored block for % ============================================================
% Paper IX — Null hypotheses and statistical protocol (RESULTS)
% Auto-authored block for % ============================================================
% Paper IX — Null hypotheses and statistical protocol (RESULTS)
% Auto-authored block for \input{../shared/paperIX_null_results.tex}
% ============================================================

\subsection{Anchored fit score used for hypothesis testing}
We use the anchored integer-lattice fit described in Secs.~\ref{sec:lattice}--\ref{sec:results} and summarize goodness-of-fit
by the mean absolute logarithmic deviation (``mean epsilon'')
\begin{equation}
\overline{\epsilon} \equiv \frac{1}{N}\sum_{i=1}^{N} \left|\log_{\varphi}\!\left(\frac{m^{(i)}_{\mathrm{pred}}}{m^{(i)}_{\mathrm{exp}}}\right)\right|.
\end{equation}
For the anchored scan on the reference set (charged leptons, $W$, $Z$, top), we obtain
\begin{equation}
\overline{\epsilon}_{\mathrm{obs}} = 6.940320\times 10^{-2}.
\end{equation}
The per-particle best-achievable $\epsilon$ values under the fixed scan box are:
\begin{center}
\begin{tabular}{l c}
\hline
Particle & $\epsilon_{\min}$ \\
\hline
electron & $0$ \\
muon     & $7.952564\times 10^{-2}$ \\
tau      & $5.529840\times 10^{-2}$ \\
$W$      & $1.161719\times 10^{-1}$ \\
$Z$      & $1.216442\times 10^{-1}$ \\
top      & $4.377904\times 10^{-2}$ \\
\hline
\end{tabular}
\end{center}

\subsection{Important note on null ensembles}
A naive ``null'' formed by permuting labels among a fixed multiset of masses is \emph{not} a valid hypothesis test for this score:
it can leave $\overline{\epsilon}$ invariant (as empirically observed in our initial permutation experiment). Therefore, all reported
$p$-values in this work will be based only on null ensembles that genuinely alter the spectrum.

\subsection{Pre-registered null models}
We pre-register two complementary null ensembles.

\paragraph{Null A (log-uniform i.i.d.).}
Draw $N$ synthetic masses independently from a log-uniform distribution on the observed range:
$\log m \sim \mathrm{Unif}(\log m_{\min},\log m_{\max})$.
Compute $\overline{\epsilon}$ using the identical scan box, anchoring convention, and tie-handling rules.

\paragraph{Null B (jittered spectrum).}
Form synthetic masses by perturbing the observed spectrum in log-space:
$\log m_i^{(\mathrm{null})}=\log m_i+\sigma\xi_i$ with $\xi_i\sim\mathcal N(0,1)$ and fixed pre-registered $\sigma$.
Compute $\overline{\epsilon}$ as above. We report results for at least two values of $\sigma$ (coarse and fine) to assess robustness.

\paragraph{Reporting rule.}
For each null, run $N_{\mathrm{null}}$ trials and report the empirical one-sided tail probability
$p_{\mathrm{emp}}=\Pr(\overline{\epsilon}_{\mathrm{null}}\le \overline{\epsilon}_{\mathrm{obs}})$ together with the null median and central intervals.
All choices (scan box, anchoring, tolerance, and null parameters) are fixed \emph{before} adding new species or extending the scan range.


% ============================================================

\subsection{Anchored fit score used for hypothesis testing}
We use the anchored integer-lattice fit described in Secs.~\ref{sec:lattice}--\ref{sec:results} and summarize goodness-of-fit
by the mean absolute logarithmic deviation (``mean epsilon'')
\begin{equation}
\overline{\epsilon} \equiv \frac{1}{N}\sum_{i=1}^{N} \left|\log_{\varphi}\!\left(\frac{m^{(i)}_{\mathrm{pred}}}{m^{(i)}_{\mathrm{exp}}}\right)\right|.
\end{equation}
For the anchored scan on the reference set (charged leptons, $W$, $Z$, top), we obtain
\begin{equation}
\overline{\epsilon}_{\mathrm{obs}} = 6.940320\times 10^{-2}.
\end{equation}
The per-particle best-achievable $\epsilon$ values under the fixed scan box are:
\begin{center}
\begin{tabular}{l c}
\hline
Particle & $\epsilon_{\min}$ \\
\hline
electron & $0$ \\
muon     & $7.952564\times 10^{-2}$ \\
tau      & $5.529840\times 10^{-2}$ \\
$W$      & $1.161719\times 10^{-1}$ \\
$Z$      & $1.216442\times 10^{-1}$ \\
top      & $4.377904\times 10^{-2}$ \\
\hline
\end{tabular}
\end{center}

\subsection{Important note on null ensembles}
A naive ``null'' formed by permuting labels among a fixed multiset of masses is \emph{not} a valid hypothesis test for this score:
it can leave $\overline{\epsilon}$ invariant (as empirically observed in our initial permutation experiment). Therefore, all reported
$p$-values in this work will be based only on null ensembles that genuinely alter the spectrum.

\subsection{Pre-registered null models}
We pre-register two complementary null ensembles.

\paragraph{Null A (log-uniform i.i.d.).}
Draw $N$ synthetic masses independently from a log-uniform distribution on the observed range:
$\log m \sim \mathrm{Unif}(\log m_{\min},\log m_{\max})$.
Compute $\overline{\epsilon}$ using the identical scan box, anchoring convention, and tie-handling rules.

\paragraph{Null B (jittered spectrum).}
Form synthetic masses by perturbing the observed spectrum in log-space:
$\log m_i^{(\mathrm{null})}=\log m_i+\sigma\xi_i$ with $\xi_i\sim\mathcal N(0,1)$ and fixed pre-registered $\sigma$.
Compute $\overline{\epsilon}$ as above. We report results for at least two values of $\sigma$ (coarse and fine) to assess robustness.

\paragraph{Reporting rule.}
For each null, run $N_{\mathrm{null}}$ trials and report the empirical one-sided tail probability
$p_{\mathrm{emp}}=\Pr(\overline{\epsilon}_{\mathrm{null}}\le \overline{\epsilon}_{\mathrm{obs}})$ together with the null median and central intervals.
All choices (scan box, anchoring, tolerance, and null parameters) are fixed \emph{before} adding new species or extending the scan range.


% ============================================================

\subsection{Anchored fit score used for hypothesis testing}
We use the anchored integer-lattice fit described in Secs.~\ref{sec:lattice}--\ref{sec:results} and summarize goodness-of-fit
by the mean absolute logarithmic deviation (``mean epsilon'')
\begin{equation}
\overline{\epsilon} \equiv \frac{1}{N}\sum_{i=1}^{N} \left|\log_{\varphi}\!\left(\frac{m^{(i)}_{\mathrm{pred}}}{m^{(i)}_{\mathrm{exp}}}\right)\right|.
\end{equation}
For the anchored scan on the reference set (charged leptons, $W$, $Z$, top), we obtain
\begin{equation}
\overline{\epsilon}_{\mathrm{obs}} = 6.940320\times 10^{-2}.
\end{equation}
The per-particle best-achievable $\epsilon$ values under the fixed scan box are:
\begin{center}
\begin{tabular}{l c}
\hline
Particle & $\epsilon_{\min}$ \\
\hline
electron & $0$ \\
muon     & $7.952564\times 10^{-2}$ \\
tau      & $5.529840\times 10^{-2}$ \\
$W$      & $1.161719\times 10^{-1}$ \\
$Z$      & $1.216442\times 10^{-1}$ \\
top      & $4.377904\times 10^{-2}$ \\
\hline
\end{tabular}
\end{center}

\subsection{Important note on null ensembles}
A naive ``null'' formed by permuting labels among a fixed multiset of masses is \emph{not} a valid hypothesis test for this score:
it can leave $\overline{\epsilon}$ invariant (as empirically observed in our initial permutation experiment). Therefore, all reported
$p$-values in this work will be based only on null ensembles that genuinely alter the spectrum.

\subsection{Pre-registered null models}
We pre-register two complementary null ensembles.

\paragraph{Null A (log-uniform i.i.d.).}
Draw $N$ synthetic masses independently from a log-uniform distribution on the observed range:
$\log m \sim \mathrm{Unif}(\log m_{\min},\log m_{\max})$.
Compute $\overline{\epsilon}$ using the identical scan box, anchoring convention, and tie-handling rules.

\paragraph{Null B (jittered spectrum).}
Form synthetic masses by perturbing the observed spectrum in log-space:
$\log m_i^{(\mathrm{null})}=\log m_i+\sigma\xi_i$ with $\xi_i\sim\mathcal N(0,1)$ and fixed pre-registered $\sigma$.
Compute $\overline{\epsilon}$ as above. We report results for at least two values of $\sigma$ (coarse and fine) to assess robustness.

\paragraph{Reporting rule.}
For each null, run $N_{\mathrm{null}}$ trials and report the empirical one-sided tail probability
$p_{\mathrm{emp}}=\Pr(\overline{\epsilon}_{\mathrm{null}}\le \overline{\epsilon}_{\mathrm{obs}})$ together with the null median and central intervals.
All choices (scan box, anchoring, tolerance, and null parameters) are fixed \emph{before} adding new species or extending the scan range.


% ============================================================

\subsection{Anchored fit score used for hypothesis testing}
We use the anchored integer-lattice fit described in Secs.~\ref{sec:lattice}--\ref{sec:results} and summarize goodness-of-fit
by the mean absolute logarithmic deviation (``mean epsilon'')
\begin{equation}
\overline{\epsilon} \equiv \frac{1}{N}\sum_{i=1}^{N} \left|\log_{\varphi}\!\left(\frac{m^{(i)}_{\mathrm{pred}}}{m^{(i)}_{\mathrm{exp}}}\right)\right|.
\end{equation}
For the anchored scan on the reference set (charged leptons, $W$, $Z$, top), we obtain
\begin{equation}
\overline{\epsilon}_{\mathrm{obs}} = 6.940320\times 10^{-2}.
\end{equation}
The per-particle best-achievable $\epsilon$ values under the fixed scan box are:
\begin{center}
\begin{tabular}{l c}
\hline
Particle & $\epsilon_{\min}$ \\
\hline
electron & $0$ \\
muon     & $7.952564\times 10^{-2}$ \\
tau      & $5.529840\times 10^{-2}$ \\
$W$      & $1.161719\times 10^{-1}$ \\
$Z$      & $1.216442\times 10^{-1}$ \\
top      & $4.377904\times 10^{-2}$ \\
\hline
\end{tabular}
\end{center}

\subsection{Important note on null ensembles}
A naive ``null'' formed by permuting labels among a fixed multiset of masses is \emph{not} a valid hypothesis test for this score:
it can leave $\overline{\epsilon}$ invariant (as empirically observed in our initial permutation experiment). Therefore, all reported
$p$-values in this work will be based only on null ensembles that genuinely alter the spectrum.

\subsection{Pre-registered null models}
We pre-register two complementary null ensembles.

\paragraph{Null A (log-uniform i.i.d.).}
Draw $N$ synthetic masses independently from a log-uniform distribution on the observed range:
$\log m \sim \mathrm{Unif}(\log m_{\min},\log m_{\max})$.
Compute $\overline{\epsilon}$ using the identical scan box, anchoring convention, and tie-handling rules.

\paragraph{Null B (jittered spectrum).}
Form synthetic masses by perturbing the observed spectrum in log-space:
$\log m_i^{(\mathrm{null})}=\log m_i+\sigma\xi_i$ with $\xi_i\sim\mathcal N(0,1)$ and fixed pre-registered $\sigma$.
Compute $\overline{\epsilon}$ as above. We report results for at least two values of $\sigma$ (coarse and fine) to assess robustness.

\paragraph{Reporting rule.}
For each null, run $N_{\mathrm{null}}$ trials and report the empirical one-sided tail probability
$p_{\mathrm{emp}}=\Pr(\overline{\epsilon}_{\mathrm{null}}\le \overline{\epsilon}_{\mathrm{obs}})$ together with the null median and central intervals.
All choices (scan box, anchoring, tolerance, and null parameters) are fixed \emph{before} adding new species or extending the scan range.

